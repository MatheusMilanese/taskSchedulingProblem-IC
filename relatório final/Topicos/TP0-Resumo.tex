%%%%%%%%%%%%%%%%%%%%%%%%
    %%% [RESUMO] %%%
%%%%%%%%%%%%%%%%%%%%%%%%
\section*{Resumo}

    Neste trabalho de conclusão de curso são propostos métodos de resolução para o problema de corte e empacotamento bidimensional com dimensões abertas. O capítulo de introdução descreve a aplicação do problema nas indústrias e as formas de solucioná-lo. No primeiro capítulo também são apresentados a hipótese de pesquisa e os objetivos do projeto. No capítulo de embasamento teórico são apresentados, de forma mais abrangente, a descrição do problema, os métodos de resolução propostos por outros autores e uma introdução as heurísticas e meta-heurísticas. No capítulo de metodologia são descritos a modelagem do problema e os meios utilizados para o desenvolvimento do projeto. No capítulo de resultados é apresentado uma discussão em cima dos resultados obtidos com a aplicação dos métodos de resolução propostos durante o estudo do problema. E no capítulo de conclusões, por fim, é apresentado uma conclusão para o trabalho realizado contendo a resposta para a hipótese de pesquisa.
    
    {\bf Palavras-chave:} Métodos exatos. Otimização. Problema de corte e empacotamento bidimensional. Strip Packing Problem 2D. Retangular. Não-guilhotinado.