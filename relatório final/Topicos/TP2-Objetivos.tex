%%%%%%%%%%%%%%%%%%%%%%%%%%%
    %%% [OBJETIVOS] %%%
%%%%%%%%%%%%%%%%%%%%%%%%%%%
\section{Objetivos}

    %%%%%%%%%%%%%%%%%%%%%%%%%%%%%%%%%%
        %%% [OBJETIVOS GERAIS] %%%
    %%%%%%%%%%%%%%%%%%%%%%%%%%%%%%%%%%
    Os objetivos gerais desse projeto incluem a apresentação de quatro métodos de resolução para o \emph{SPP-2D} com dimensões abertas, considerando o problema de corte não-guilhotinado de um objeto retangular, sem a rotação ortogonal (Rotação de 90 graus) de itens retangulares, e a realização de uma análise de desempenho, a partir dos resultados obtidos com a resolução de várias instâncias do problema existentes na literatura, para identificar o método que encontra, no menor tempo de execução, a melhor configuração dos itens no objeto.
    
    
    %%%%%%%%%%%%%%%%%%%%%%%%%%%%%%%%%%%%%%%
        %%% [OBJETIVOS ESPECIFICOS] %%%
    %%%%%%%%%%%%%%%%%%%%%%%%%%%%%%%%%%%%%%%
    \newpage
    Os objetivos específicos da pesquisa são:
    
    \begin{itemize} %[align = parleft, left = 50pt..6em]
        \item Apresentar um modelo matemático para o problema, com quatro abordagens de resolução.
        \item Discutir o desempenho dos métodos desenvolvidos na resolução do problema.
        \item Apresentar os resultados do projeto respondendo à hipótese de pesquisa.
    \end{itemize}